\documentclass{beamer}
\usepackage{amsmath}
\usepackage{color}
\usepackage{cancel}

\title{The Low-Index Subgroups Algorithm}
\subtitle{Approaches to parallelisation in HPC-GAP}
\author{Michael Torpey}
\institute{University of St Andrews}
\date{22nd August 2013}

\renewcommand\mathfamilydefault{\rmdefault}
\usecolortheme{albatross}
\setbeamertemplate{navigation symbols}{}

\begin{document}

\maketitle

\begin{frame}
  \frametitle{The question}
  Given a finitely presented group $G = \langle X | R \rangle$, what are its subgroups of index no more than $N$?
  \begin{itemize}
  \item<2-> $X = $ A set of generators, e.g.~$\{a,b\}$.
  \item<3-> $R = $ A set of relators, e.g.~$\{a^2,b^3,(ab)^5\}$ such that $a^2=b^3=(ab)^5=1$.
  \item<4-> $G = \langle a,b | a^2=b^3=(ab)^5=1\rangle \cong A_5$
  \end{itemize}
\end{frame}

\begin{frame}
  \frametitle{The algorithm}
  \begin{itemize}
  \item ``Forced coincidence'' approach
  \item Utilises Todd-Coxeter method for coset enumeration
  \item Expand coset table defining no more than $n$ cosets, for some $n \geq N$.
  \end{itemize}
\end{frame}

\begin{frame}[t]
  \frametitle{Coset enumeration}
  \pause
  Todd-Coxeter algorithm:
  \pause
  \begin{table}
    \begin{tabular}{r | c c c c}
      & $a$ & $a^{-1}$ & $b$ & $b^{-1}$ \\ \hline
      $\phantom{b^{-1}}H=1$ & & & \phantom{3} & \\ \\ \\ \\
    \end{tabular}
  \end{table}

\end{frame}

\begin{frame}[t]
  \frametitle{Coset enumeration}
  Todd-Coxeter algorithm:
  \begin{table}
    \begin{tabular}{r | c c c c}
      & $a$ & $a^{-1}$ & $b$ & $b^{-1}$ \\ \hline
      $\phantom{b^{-1}}H=1$ & 2 & & \phantom{3} & \\
      $Ha=2$ & & 1 & & \\ \\ \\
    \end{tabular}
  \end{table}

  \begin{itemize}
  \item SET $1^a=2$
  \end{itemize}
\end{frame}

\begin{frame}[t]
  \frametitle{Coset enumeration}
  Todd-Coxeter algorithm:
  \begin{table}
    \begin{tabular}{r | c c c c}
      & $a$ & $a^{-1}$ & $b$ & $b^{-1}$ \\ \hline
      $\phantom{b^{-1}}H=1$ & 2 & 2 & \phantom{3} & \\
      $Ha=2$ & 1 & 1 & & \\ \\ \\
    \end{tabular}
  \end{table}

  \begin{itemize}
  \item SET $1^a=2$  
  \item SCAN $a^2$ on coset $1$: \\ $1 \overset{a}{\rightarrow} 2 \overset{a}{\rightarrow} 1$
  \end{itemize}
\end{frame}

\begin{frame}[t]
  \frametitle{Coset enumeration}
  Todd-Coxeter algorithm:
  \begin{table}
    \begin{tabular}{r | c c c c}
      & $a$ & $a^{-1}$ & $b$ & $b^{-1}$ \\ \hline
      $\phantom{b^{-1}}H=1$  & 2 & 2 & 3 &   \\
      $Ha=2$ & 1 & 1 &   &   \\
      $Hb=3$ &   &   &   & 1 \\ \\
    \end{tabular}
  \end{table}

  \begin{itemize}
  \item SET $1^a=2$  
  \item SCAN $a^2$ on coset $1$: \\ $ 1 \overset{a}{\rightarrow} 2 \overset{a}{\rightarrow} 1$
  \item SET $1^b = 3$
  \end{itemize}
\end{frame}

\begin{frame}[t]
  \frametitle{Coset enumeration}
  Todd-Coxeter algorithm:
  \begin{table}
    \begin{tabular}{r | c c c c}
      & $a$ & $a^{-1}$ & $b$ & $b^{-1}$ \\ \hline
      $H=1$  & 2 & 2 & 3 & 4 \\
      $Ha=2$ & 1 & 1 &   &   \\
      $Hb=3$ &   &   &   & 1 \\
      $Hb^{-1}=4$ && & 1 &
    \end{tabular}
  \end{table}

  \begin{itemize}
  \item SET $1^a=2$  
  \item SCAN $a^2$ on coset $1$: \\ $ 1 \overset{a}{\rightarrow} 2 \overset{a}{\rightarrow} 1$
  \item SET $1^b = 3$
  \item SET $1^b = 4$
  \end{itemize}
\end{frame}

\begin{frame}[t]
  \frametitle{Coset enumeration}
  Todd-Coxeter algorithm:
  \begin{table}
    \begin{tabular}{r | c c c c}
      & $a$ & $a^{-1}$ & $b$ & $b^{-1}$ \\ \hline
      $H=1$  & 2 & 2 & 3 & 4 \\
      $Ha=2$ & 1 & 1 &   &   \\
      $Hb=3$ &   &   & 4 & 1 \\
      $Hb^{-1}=4$ && & 1 & 3
    \end{tabular}
  \end{table}

  \begin{itemize}
  \item SET $1^a=2$  
  \item SCAN $a^2$ on coset $1$: \\ $ 1 \overset{a}{\rightarrow} 2 \overset{a}{\rightarrow} 1$
  \item SET $1^b = 3$
  \item SET $1^b = 4$
  \item SCAN $b^3$ on coset $4$: \\ $ 4 \overset{b}{\rightarrow} 1 \overset{b}{\rightarrow} 3 \overset{b}{\rightarrow} 4$
  \end{itemize}
\end{frame}

\begin{frame}
  \frametitle{Coincidences}
  Sometimes we may encounter a coincidence. \pause

  Example:
  \begin{table}
    \begin{tabular}{r | c c c c}
      & $a$ & $a^{-1}$ \\ \hline
      1 & 2 & \\
      2 & \phantom{1} 3 \phantom{1} & 1 \\
      3 &   & 2
    \end{tabular}
  \end{table}
  \pause
  \begin{itemize}
  \item SCAN $a^2$ on coset $1$ \pause
  \item $ 1 \overset{a}{\rightarrow} 2 \overset{a}{\rightarrow} 3$ \pause
  \item But we should have $ 1 \overset{a}{\rightarrow} \overset{a}{\rightarrow} 1$ \pause
  \item Hence $1$ and $3$ describe the same coset, and they can be combined
  \end{itemize}
\end{frame}

\begin{frame}
  \frametitle{Coincidences}
  Sometimes we may encounter a coincidence.

  Example:
  \begin{table}
    \begin{tabular}{r | c c c c}
      & $a$ & $a^{-1}$ \\ \hline
      1 & 2 & {\color{red}2}\\
      2 & \phantom{1} \cancel{3} {\color{red}1} & 1 \\
      \cancel{3} &   & \cancel{2}
    \end{tabular}
  \end{table}
  
  \begin{itemize}
  \item SCAN $a^2$ on coset $1$ 
  \item $ 1 \overset{a}{\rightarrow} 2 \overset{a}{\rightarrow} 3$ 
  \item But we should have $ 1 \overset{a}{\rightarrow} \overset{a}{\rightarrow} 1$ 
  \item Hence $1$ and $3$ describe the same coset, and they can be combined
  \end{itemize}
\end{frame}

\begin{frame}
  \frametitle{Coincidences}
  Sometimes we may encounter a coincidence.

  Example:
  \begin{table}
    \begin{tabular}{r | c c c c}
      & $a$ & $a^{-1}$ \\ \hline
      1 & 2 & 2\\
      2 & \phantom{3} 1 \phantom{1} & 1 \\ \\
    \end{tabular}
  \end{table}
  
  \begin{itemize}
  \item SCAN $a^2$ on coset $1$ 
  \item $ 1 \overset{a}{\rightarrow} 2 \overset{a}{\rightarrow} 3$ 
  \item But we should have $ 1 \overset{a}{\rightarrow} \overset{a}{\rightarrow} 1$ 
  \item Hence $1$ and $3$ describe the same coset, and they can be combined
  \end{itemize}
\end{frame}

\begin{frame}
  \frametitle{Forcing a coincidence}
  \begin{itemize}
  \item Eventually we cannot continue because either: \pause
    \begin{itemize}
    \item The coset table is complete\pause, or
    \item We have defined $n$ cosets, the maximum number \pause
    \end{itemize}
  \item If the table is complete, we have a subgroup \pause
  \item In any case, we now ``force a coincidence'' \pause
  \item Take some pair of cosets $i$ and $j$, and force $i=j$ \pause
  \item The resultant table now corresponds to a new subgroup with a new generator $\alpha_i \alpha_j^{-1}$ constructed from the coset representatives $\alpha_i$ and $\alpha_j$ \pause
  \item Each choice of $(i,j)$ is considered separately as a new branch in the search tree
  \end{itemize}
\end{frame}

\begin{frame}
  \frametitle{Characteristics}
  We have a backtrack search that: \pause
  \begin{itemize}
  \item is unpredictable in shape \pause
  \item is unpredictable in size \pause
  \item may return results before reaching a leaf \pause
  \item can be split into independent branches
  \end{itemize}
\end{frame}

\begin{frame}
  \frametitle{Parallelisation}
  Two approaches taken:
  \begin{itemize}
  \item Tasks (using \texttt{RunTask}, \texttt{WaitTask}\dots)
  \item Worker threads (\texttt{CreateThread}, \texttt{WaitThread}\dots)
  \end{itemize}
\end{frame}

\begin{frame}[fragile]
  \frametitle{Sequential implementation}
  \framesubtitle{Recursion}
\begin{verbatim}
DescendantSubgroups := function(...)
    subgps := [];
    CosetEnumeration(...);
    if IsComplete(table) then
        Add(subgps, thisSubgroup);
    fi;
    for each pair of cosets (i,j) do
        Append(subgps,
               DescendantSubgroups(<table with i=j>, ...)
              );
    od;
    return subgps;
end;
\end{verbatim}
\end{frame}

\begin{frame}[fragile]
  \frametitle{Using Tasks}
  \framesubtitle{The ``shotgun'' approach}
\begin{verbatim}
DescendantSubgroups := function(...)
    subgps := [];
    tasks := [];
    CosetEnumeration(...);
    if IsComplete(table) then
        Add(subgps, thisSubgroup);
    fi;
    for each pair of cosets (i,j) do
        Add(tasks, RunTask(DescendantSubgroups, <args>) );
    od;
    for task in tasks do
        Append(subgps, TaskResult(task) );
    od;
    return subgps;
end;
\end{verbatim}
\end{frame}

\begin{frame}
  \frametitle{Using Tasks}
  \framesubtitle{Speedup}
  \begin{itemize}
  \item Effective up to 4 cores \pause
  \item Little speedup beyond 4 cores \pause
  \item Enormous time for large problems -- overheads
  \end{itemize}
\end{frame}

\begin{frame}
  \frametitle{Using Worker Threads}
  \framesubtitle{Objects} \pause
  \begin{itemize}
  \item \texttt{workQueue} -- Channel of jobs to be done \pause
  \item \texttt{numJobs} -- Number of jobs still incomplete \pause
  \item \texttt{resultsChan} -- Channel used to store results \pause
  \item \texttt{finish} -- Semaphore indicating that all work is complete \pause
  \item \texttt{Work} -- Function executed by each new thread \pause
  \item \texttt{ExecuteJob} -- New name for \texttt{DescendantSubgroups}
  \end{itemize}
\end{frame}

\begin{frame}[fragile]
  \frametitle{Using Worker Threads}
  \framesubtitle{Top-level function}
\begin{verbatim}
LowIndexSubgroups(G, maxIndex, numWorkers)
    ...
    <Create workQueue, resultsChan, numJobs, and finish>

    workers := List([1..numWorkers],
                    i->CreateThread(Work, <args>)
                   );
    ExecuteJob(..., workQueue, resultsChan, numJobs);

    WaitSemaphore(finish);
    SendChannel(workQueue, fail);
    Perform(workers, WaitThread);

    <Extract all the results from resultsChan>
    ...
end;
\end{verbatim}
\end{frame}

\begin{frame}[fragile]
  \frametitle{Using Worker Threads}
  \framesubtitle{\texttt{Work} function}
\begin{verbatim}
Work := function(workQueue, resultsChan, ...)
    while true do
        j := ReceiveChannel(workQueue);
        if j = fail then
            SendChannel(workQueue,fail);
            break;
        fi;
        ExecuteJob(j.table, j.label, ...);
        atomic numJobs do
            numJobs := numJobs - 1;
            if numJobs = 0 then
                SignalSemaphore(finish);
            fi;
        od;
    od;
end;
\end{verbatim}
\end{frame}

\begin{frame}[fragile]
  \frametitle{Using Worker Threads}
  \framesubtitle{\texttt{ExecuteJob} function}
\begin{verbatim}
ExecuteJob := function(...)
    CosetEnumeration(...);
    if IsComplete(table) then
        SendChannel(resultsChan, thisSubgroup);
    fi;
    for each pair of cosets (i,j) do
        newJob := rec(table := table,
                      label := b,
                      reps := reps,
                      gens := Concatenation(gens,[newGen])
                     );
        SendChannel(workQueue, newJob);
        atomic numJobs do
            numJobs := numJobs + 1;
        od;
    od;
end;
\end{verbatim}
\end{frame}

\begin{frame}
  \frametitle{Using Worker Threads}
  \framesubtitle{Speedup} \pause
  \begin{itemize}
  \item Effective up to 4 cores
  \item Little speedup beyond 4 cores \pause
  \item Huge number of jobs created -- all threads attempting to read from workQueue very often, resulting in a bottleneck \pause
  \item If only workers could explore subtrees themselves, so long as all cores are busy...
  \end{itemize}
\end{frame}

\begin{frame}[fragile]
  \frametitle{``Minimal'' Job Sharing} \pause
  \begin{itemize}
  \item If every thread has work to do, a thread processes a complete job depth-first with no communication \pause
  \item If there is no work left on the queue, a thread must branch \pause
  \item Avoids either heavy communication on a single channel, or long-idle workers \pause
  \item New parameter in \texttt{ExecuteJob} -- \texttt{depthFirst} \pause
  \end{itemize}
In the \texttt{Work} function:
\begin{verbatim}
atomic readonly numJobs do
    depthFirst := numJobs > numWorkers;
od;
\end{verbatim} \pause
\begin{itemize}
\item Still have workers idle, waiting for another thread to branch
\end{itemize}

\end{frame}

\begin{frame}
  \frametitle{Improvements}
  \begin{itemize}
  \item Decide whether to branch \textit{inside} depth-first search \pause
  \item Always keep a ``buffer'' of items on the queue, to reduce idle workers -- means more breadth-first \pause
  \item Attempt to predict size of subtree and ``branch intelligently'' \pause
  \end{itemize}

  Other approaches:
  \begin{itemize}
  \item Retrospective job sharing \pause
  \item Random depth-first search
  \end{itemize}
\end{frame}

\end{document}
