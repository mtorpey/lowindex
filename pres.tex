\documentclass{beamer}
\usepackage{amsmath}
\usepackage{color}
\usepackage{cancel}

\title{The Low-Index Subgroups Algorithm}
\subtitle{Approaches to parallelisation in HPC-GAP}
\author{Michael Torpey}
\institute{University of St Andrews}
\date{21st August 2013}

\renewcommand\mathfamilydefault{\rmdefault}
\usecolortheme{albatross}
\setbeamertemplate{navigation symbols}{}

\begin{document}

\maketitle

\begin{frame}
  \frametitle{The question}
  Given a finitely presented group $G = \langle X | R \rangle$, what are its subgroups of index no more than $N$?
  \begin{itemize}
  \item<2-> $X = $ A set of generators, e.g.~$\{a,b\}$.
  \item<3-> $R = $ A set of relators, e.g.~$\{a^2,b^3,(ab)^5\}$ such that $a^2=b^3=(ab)^5=1$.
  \item<4-> $G = \{a,b | a^2=b^3=(ab)^5=1\} \cong A_5$
  \end{itemize}
\end{frame}

\begin{frame}
  \frametitle{The algorithm}
  \begin{itemize}
  \item ``Forced coincidence'' approach
  \item Utilises Todd-Coxeter method for coset enumeration
  \item Expand coset table defining no more than $n$ cosets, for some $n \geq N$.
  \end{itemize}
\end{frame}

\begin{frame}[t]
  \frametitle{Coset enumeration}
  \pause
  Todd-Coxeter algorithm:
  \pause
  \begin{table}
    \begin{tabular}{r | c c c c}
      & $a$ & $a^{-1}$ & $b$ & $b^{-1}$ \\ \hline
      $\phantom{b^{-1}}H=1$ & & & \phantom{3} & \\ \\ \\ \\
    \end{tabular}
  \end{table}

\end{frame}

\begin{frame}[t]
  \frametitle{Coset enumeration}
  Todd-Coxeter algorithm:
  \begin{table}
    \begin{tabular}{r | c c c c}
      & $a$ & $a^{-1}$ & $b$ & $b^{-1}$ \\ \hline
      $\phantom{b^{-1}}H=1$ & 2 & & \phantom{3} & \\
      $Ha=2$ & & 1 & & \\ \\ \\
    \end{tabular}
  \end{table}

  \begin{itemize}
  \item SET $1^a=2$
  \end{itemize}
\end{frame}

\begin{frame}[t]
  \frametitle{Coset enumeration}
  Todd-Coxeter algorithm:
  \begin{table}
    \begin{tabular}{r | c c c c}
      & $a$ & $a^{-1}$ & $b$ & $b^{-1}$ \\ \hline
      $\phantom{b^{-1}}H=1$ & 2 & 2 & \phantom{3} & \\
      $Ha=2$ & 1 & 1 & & \\ \\ \\
    \end{tabular}
  \end{table}

  \begin{itemize}
  \item SET $1^a=2$  
  \item SCAN $a^2$ on coset $1$: \\ $1 \overset{a}{\rightarrow} 2 \overset{a}{\rightarrow} 1$
  \end{itemize}
\end{frame}

\begin{frame}[t]
  \frametitle{Coset enumeration}
  Todd-Coxeter algorithm:
  \begin{table}
    \begin{tabular}{r | c c c c}
      & $a$ & $a^{-1}$ & $b$ & $b^{-1}$ \\ \hline
      $\phantom{b^{-1}}H=1$  & 2 & 2 & 3 &   \\
      $Ha=2$ & 1 & 1 &   &   \\
      $Hb=3$ &   &   &   & 1 \\ \\
    \end{tabular}
  \end{table}

  \begin{itemize}
  \item SET $1^a=2$  
  \item SCAN $a^2$ on coset $1$: \\ $ 1 \overset{a}{\rightarrow} 2 \overset{a}{\rightarrow} 1$
  \item SET $1^b = 3$
  \end{itemize}
\end{frame}

\begin{frame}[t]
  \frametitle{Coset enumeration}
  Todd-Coxeter algorithm:
  \begin{table}
    \begin{tabular}{r | c c c c}
      & $a$ & $a^{-1}$ & $b$ & $b^{-1}$ \\ \hline
      $H=1$  & 2 & 2 & 3 & 4 \\
      $Ha=2$ & 1 & 1 &   &   \\
      $Hb=3$ &   &   &   & 1 \\
      $Hb^{-1}=4$ && & 1 &
    \end{tabular}
  \end{table}

  \begin{itemize}
  \item SET $1^a=2$  
  \item SCAN $a^2$ on coset $1$: \\ $ 1 \overset{a}{\rightarrow} 2 \overset{a}{\rightarrow} 1$
  \item SET $1^b = 3$
  \item SET $1^b = 4$
  \end{itemize}
\end{frame}

\begin{frame}[t]
  \frametitle{Coset enumeration}
  Todd-Coxeter algorithm:
  \begin{table}
    \begin{tabular}{r | c c c c}
      & $a$ & $a^{-1}$ & $b$ & $b^{-1}$ \\ \hline
      $H=1$  & 2 & 2 & 3 & 4 \\
      $Ha=2$ & 1 & 1 &   &   \\
      $Hb=3$ &   &   & 4 & 1 \\
      $Hb^{-1}=4$ && & 1 & 3
    \end{tabular}
  \end{table}

  \begin{itemize}
  \item SET $1^a=2$  
  \item SCAN $a^2$ on coset $1$: \\ $ 1 \overset{a}{\rightarrow} 2 \overset{a}{\rightarrow} 1$
  \item SET $1^b = 3$
  \item SET $1^b = 4$
  \item SCAN $b^3$ on coset $4$: \\ $ 4 \overset{b}{\rightarrow} 1 \overset{b}{\rightarrow} 3 \overset{b}{\rightarrow} 4$
  \end{itemize}
\end{frame}

\begin{frame}
  \frametitle{Coincidences}
  Sometimes we may encounter a coincidence. \pause

  Example:
  \begin{table}
    \begin{tabular}{r | c c c c}
      & $a$ & $a^{-1}$ \\ \hline
      1 & 2 & \\
      2 & \phantom{1} 3 \phantom{1} & 1 \\
      3 &   & 2
    \end{tabular}
  \end{table}
  \pause
  \begin{itemize}
  \item SCAN $a^2$ on coset $1$ \pause
  \item $ 1 \overset{a}{\rightarrow} 2 \overset{a}{\rightarrow} 3$ \pause
  \item But we should have $ 1 \overset{a}{\rightarrow} \overset{a}{\rightarrow} 1$ \pause
  \item Hence $1$ and $3$ describe the same coset, and they can be combined
  \end{itemize}
\end{frame}

\begin{frame}
  \frametitle{Coincidences}
  Sometimes we may encounter a coincidence.

  Example:
  \begin{table}
    \begin{tabular}{r | c c c c}
      & $a$ & $a^{-1}$ \\ \hline
      1 & 2 & {\color{red}2}\\
      2 & \phantom{1} \cancel{3} {\color{red}1} & 1 \\
      \cancel{3} &   & \cancel{2}
    \end{tabular}
  \end{table}
  
  \begin{itemize}
  \item SCAN $a^2$ on coset $1$ 
  \item $ 1 \overset{a}{\rightarrow} 2 \overset{a}{\rightarrow} 3$ 
  \item But we should have $ 1 \overset{a}{\rightarrow} \overset{a}{\rightarrow} 1$ 
  \item Hence $1$ and $3$ describe the same coset, and they can be combined
  \end{itemize}
\end{frame}

\begin{frame}
  \frametitle{Coincidences}
  Sometimes we may encounter a coincidence.

  Example:
  \begin{table}
    \begin{tabular}{r | c c c c}
      & $a$ & $a^{-1}$ \\ \hline
      1 & 2 & 2\\
      2 & \phantom{3} 1 \phantom{1} & 1 \\ \\
    \end{tabular}
  \end{table}
  
  \begin{itemize}
  \item SCAN $a^2$ on coset $1$ 
  \item $ 1 \overset{a}{\rightarrow} 2 \overset{a}{\rightarrow} 3$ 
  \item But we should have $ 1 \overset{a}{\rightarrow} \overset{a}{\rightarrow} 1$ 
  \item Hence $1$ and $3$ describe the same coset, and they can be combined
  \end{itemize}
\end{frame}

\begin{frame}
  \frametitle{Forcing a coincidence}
  \begin{itemize}
  \item Eventually we cannot continue because either:
    \begin{itemize}
    \item The coset table is complete\pause, or
    \item We have defined $n$ cosets, the maximum number
    \end{itemize}
  \item If the table is complete, we have a subgroup \pause
  \item In any case, we now ``force a coincidence''
  \end{itemize}
\end{frame}

\begin{frame}
  \frametitle{Forcing a coincidence}
  \begin{itemize}
  \item Take some pair of cosets $i$ and $j$, and force $i=j$ \pause
  \item The resultant table now corresponds to a new subgroup with a new generator $\alpha_i \alpha_j^{-1}$ constructed from the coset representatives $\alpha_i$ and $\alpha_j$ \pause
  \item Each choice of $(i,j)$ is considered separately as a new branch in the search tree \pause
  \item To avoid duplicates, each new branch is assigned a ``label'', the higher of the combined cosets \pause
  \item We only consider coincidences for which $i$ and $j$ are not both less than \texttt{label}
  \end{itemize}
\end{frame}

\begin{frame}
  \frametitle{Characteristics}
  We have a backtrack search that: \pause
  \begin{itemize}
  \item is unpredictable in shape \pause
  \item is unpredictable in size \pause
  \item may return results before reaching a leaf \pause
  \item can be split into independent branches
  \end{itemize}
\end{frame}

\begin{frame}
  \frametitle{Parallelisation}

\end{frame}

\end{document}